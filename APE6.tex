\newape{DSL Systems}

\begin{solution}
	\begin{enumerate}
		\item We have 2048 carriers, spaced of $4.3125kHz$, we can then say that the total bandwidth is : 
			  $$BW = 2048\cdot 4.3125kHz = 8.832 Mhz$$
		\item The FFT must be supplied by an input of size power of 2, with the positive and negative frequencies. So we have a size of $2\cdot 2048 = 4096$.
		\item The total bandwidth is $8.832 Mhz$ but because of the Shannon theorem, we have to use a frequency higher than two times the bandwidth. So we will use a frequency $f_s \geq 17.664Mhz$.
		\item The length is the output of the $IFFT + |CP| = 4096 + 320$ The duration is $T_s = N_s \cdot dt = N_s/f_s = \frac{4096 + 320}{17.664\expten{6}} = 25\expten{-4} = 25ms$.
		\item The SNR available depends on the frequency because of the non-flat frequency response of the channel. The formula given in the exercice seems wrong to me for two reasons:
		\begin{itemize}
			\item The units doesn't match because of the square
			\item The values of attenuation resulting are too big with the square...
		\end{itemize}				
		
		$$
		SNR = \frac{\gamma_s \lambda}{\gamma_n}\frac{2N}{2N + L} = \frac{1\expten{-6} (1\expten{-att/10})}{1\expten{-14}}\frac{2\cdot 2048}{2\cdot 2048 + 320}
		$$
		Beware that the PSD are given in dBm, so we have to convert to dB. This doesn't change the answer because those values are divided but it's a great reflex to have.
		
		
		Which gives us: $SNR_{1} = 11676.98 = 40.6733 dB$, $SNR_{2} = 92.75 = 19.67 dB$, $SNR_{3} = 2.933 = 4.67 dB$.
		
		We want a SNR margin of $6dB$ and we know that the coding allows to gain $3dB$, so we can say that we have to compute $SNR_e = SNR - 6 + 3 = SNR - 3$.
		Then we have $SNR'_{1} = 37.6733 dB$, $SNR'_{2} = 16.67 dB$, $SNR'_{3} = 1.67 dB$.
		
		If we match those SNR in the plot of BER/SNR given, we can choose the modulation which will be: 64-QAM fo the frequencies from 50kHz to $3MHz$, 4-QAM for the frequencies between $3MHz$ and $6MHz$ and BPSK for higher ones.
		Those modulations conduct to a bitrate of $N_1\cdot log_2 64 + N_2\cdot log_2 4 + N_3\cdot 1$ where $N_i$ corresponds to the amount of carriers in each frequency range. 
		$N_1 = (3Mhz - 50kHz)/4.3125k = 684$, $N_1 = 3Mhz/4.3125k = 695$, $N_3 = 2048 - 684 - 695 = 669$. Then, we can simply sum and multiply everything:
		
		
		$BR = 6163/25ms = 246520 bits/s$ which seems wrong to me because VDSL is designed to have higher bitrates than ADSL... \notsure	
						
	\end{enumerate}
\end{solution}

\begin{solution}
	We know the representation of the line. It's a self with a resistance in serial, and also a capacity with another resistance in parallel between the two wires.
	Then, we can compute the quadripole representation of the different lines here. We have 3 pieces in total: $d_1$, $d_{BT}$ and $d_2$. We compute the S matrix corresponding to those quadripoles.
	$$
	\left[
	\begin{array}{c}
	V_I \\ 
	I_I
	\end{array}
	\right] 
	= 
	\phi
	\left[
	\begin{array}{c}
	V_L \\ 
	I_L
	\end{array}
	\right] 
	$$
	
	Normally, $\phi$ is given, to recompute it, it's long and not useful, so we have 
	$$\phi_{connected} =
	\left[
	\begin{array}{cc}
	cosh(\gamma d) & Z_0 sinh(\gamma d) \\ 
	\frac{sinh(\gamma d)}{Z_0} & cosh(\gamma d)
	\end{array} 
	\right]$$	
	with $\gamma = \sqrt{(R + j\omega L)(G + j \omega C)}$
	
	The matrix for the disconnected tape (open) is :
	$$\phi_{open} =
	\left[
	\begin{array}{cc}
	1 & 0 \\ 
	\frac{-tanh(\gamma d)}{Z_{BT}} & 1
	\end{array} 
	\right]$$	
	
	Then we can use the total quadripole which is equivalent to $\phi = \phi_{connected 1} \phi{open} \phi{connected 2}$. The transfer function is then given by $\frac{V_L}{V_I} = \frac{1}{\phi_{11} + \phi_{12}/Z}$ where $Z$ is load corresponding to the user.
	
	After the computation (very long, boring and useless, if someone want to copy it, he can !), we obtain: 
	$$T = \frac{V_L}{V_I} = \frac{e^{-\gamma d}}{1 + \frac{1}{2}tanh(\gamma d_{BT})}$$
\end{solution}